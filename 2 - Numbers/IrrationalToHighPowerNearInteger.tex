\subsection{Irrational to High Power Near Integer}

This problem was given to me by my Oxford admissions preparation tutor, Jeremy. The original problem he gave was ``what happens, now prove it", where I found the first part, and which I have decided to reveal upfront. I came back to this problem in 2024 and solved it in around 20 minutes and I have provided this proof as the second solution. Dr Barker on YouTube also covered this problem and gave the much slicker argument which is given as the first solution. This problem benefits greatly from pen and paper, and is much more suited to the mathematically inclined. I rate this problem a 5 out of 10 in terms of difficulty.

$(1 + \sqrt{2})^n$ gets close to an integer when raised to a high power. Explain why this is the case. Extension question: characterise exactly which $a + \sqrt{b}$ where this occurs.

\textbf{Hints for Solution 1:}

\begin{enumerate}
    \item Use the binomial expansion to expand $\left( 1 + \sqrt{2} \right)^n$. How would you need to modify this to make the odd powers of $\sqrt{2}$ disappear?
    \item Consider the conjugate, $\left( 1 - \sqrt{2} \right)^n$.
    \item Add the conjugate to the original to make the odd powers cancel out.
\end{enumerate}

\textbf{Solution 1:}

We define $a_n = \left( 1 + \sqrt{2} \right)^n$ and $b_n$ as the sum of $\left (1 + \sqrt{2} \right)^n$ and its conjugate. Using the binomial expansion we see in equation~\eqref{eqn:IrrationalToHighPowerNearInteger} that the odd powers of $\sqrt{2}$ cancel out, and thus $b_n$ is an integer. We also note that $\left| 1 - \sqrt{2} \right| < 1$, and therefore it annihilates when raising it to increasingly higher powers. This means $|a_n - b_n| \to 0$ as $n \to \infty$, and thus $a_n$ converges to an integer.

\begin{align}
	\begin{split}
		b_n &= \left( 1 + \sqrt{2} \right)^n + \left( 1 - \sqrt{2} \right)^n  \\
		&= \sum_{k=0}^n {n \choose k} \left( 1^{n - k} \cdot \sqrt{2}^k + 1^{n - k} \cdot (-\sqrt{2})^k \right)  \\
		&= \sum_{k=0}^n {n \choose k} \sqrt{2}^k \cdot (1 + (-1)^k)  \\
		&= \sum_{k=0}^n {n \choose k} \sqrt{2}^k \cdot \{k \ \text{is even}\}  \\
		&= \sum_{m=0}^{\left\lfloor \frac{n}{2} \right\rfloor} {n \choose 2m} 2^m \in \mathbb{N}
		\label{eqn:IrrationalToHighPowerNearInteger}
	\end{split}	
\end{align}

\textbf{Hints for Solution 2:}

\begin{enumerate}
    \item 
\end{enumerate}

\textbf{Solution 2:}



\textbf{Extensions and Comments:}