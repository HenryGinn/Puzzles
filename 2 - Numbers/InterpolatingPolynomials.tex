\subsection{Interpolating Polynomials}

This subsection is about a series of results I found about interpolating polynomials that are all linked together. I found these at several different points throughout my mathematical journey, and have attempted to present them in a way that highlights how one might discover them for themselves.

\subsubsection{Calculus of Differences and $n^\text{th}$ Term Rule}

The $n^\text{th}$ term rule is a function from the natural numbers excluding 0 to the reals that generates a sequence. In year 9 we were shown a method to find the $n^\text{th}$ term rule for a sequence generated by a quadratic polynomial that confused me, and when I investigated this for myself I found an extension to higher orders. I did not prove that this always worked at the time, and did so when I revisited the topic during my maths masters. Determining a cubic polynomial for a four term sequence was one of the very first programs I wrote soon after I turned 15. During the winter holiday I wrote a version that could find the polynomial for arbitrary length sequences.

\subsubsection{The Lagrange Interpolating Polynomial}

In my second year of undergrad I was doing problem sheet 1 from part A linear algebra~\cite{PartALinearAlgebraSheet1}, and question two was the following.

\begin{center}
    (Harder) Show that the space of functions $f : \mathbb{N} \to \mathbb{R}$ does not have a countable basis.
\end{center}

I was not initially convinced by this and thought that a polynomial interpolating all the points of $f$ would be a counterexample. I have discussed why this reasoning is wrong in the extensions and comments, along with the answer to the original question. I am also not sure why I did not realise I had my supposed counterexample from the calculus of differences method, but nonetheless I found new a method that is the subject of this exercise. If I was a teacher then I would definitely be giving this problem to my students.

Given a finite set, $A \subset \mathbb{R}$, and a function $f : A \to \mathbb{R}$, find an interpolating polynomial $p : A \to \mathbb{R}$. This is the polynomial satisfying $f(x_i) = p(x_i) \quad \forall x_i \in A$, and the polynomial of minimal degree satisfying this is known as the Lagrange Interpolating Polynomial, or simply the Lagrange Polynomial.

\textbf{Hints:}

\begin{enumerate}
    \item 
\end{enumerate}

\textbf{Solution:}



\textbf{Extensions and Comments:}

% Give some more background on why this subsection is called the Lagrange interpolating polynomial - the question itself never specifies that this is of minimal order.
% Why is this not a counter example?
% How do you actually prove the statement?
% Discuss the numerical instability of the polynomial as presented here

\subsubsection{Properties of Resulting Polynomials}

% Minimality of polynomials
% Uniqueness of minimal interpolating polynomial through Vandermonde determinant
% Combining these properties to show equality from the two methods