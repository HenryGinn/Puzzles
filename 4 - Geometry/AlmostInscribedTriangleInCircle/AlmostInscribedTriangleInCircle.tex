\subsection{Almost Inscribed Triangle In Circle}

I found this from Catriona Agg's collection of geometry problems and it appealed to me as it seemed very minimal. The first solution was found by me and the second simpler solution was found by my dad. I rate this problem a 4 out of 10 in terms of difficulty.

The triangle $ABC$ is equilateral and $|AD| = 9$. Find the area of the circle.

\begin{center}
	\begin{tikzpicture}[scale=3]
		% Defining coordinates
		\coordinate (A) at (0.25, 0.9682458365518544);
\coordinate (B) at (-0.7135254915624212, -0.7006292692220368);
\coordinate (C) at (1.2135254915624212, -0.7006292692220368);
\coordinate (D) at (0.7135254915624212, -0.7006292692220368);
\coordinate (E) at (-0.9635254915624212, -0.26761656732981753);
\coordinate (F) at (0.9635254915624212, 0.26761656732981753);
\coordinate (a) at (0.4817627457812106, 0.13380828366490877);
\coordinate (b) at (0.3658813728906053, 0.5510270601083815);
\coordinate (c) at (0.125, 0.4841229182759272);
\coordinate (d) at (0.5590169943749475, -0.14433756729740643);
		
		\node at (A) [above = 0.1 of A] {$A$};
		\node at (B) [below left = 0.1 of B] {$B$};
		\node at (C) [below right = 0.1 of C] {$C$};
		\node at (D) [below right = 0.1 of D] {$D$};
		\node at (a) [left = 0.2 of a] {$9$};
		
		\draw (0, 0) circle[radius=1];
		\draw (A) -- (B);
		\draw (A) -- (C);
		\draw (A) -- (D);
		\draw (B) -- (C);
	\end{tikzpicture}
\end{center}

\textbf{Hints 1:}

\begin{enumerate}
	\item The triangle $ADC$ contains segment $AC$ which is equal in length to segment $AB$. Use this to construct a congruent triangle.
	\item Prove that the added point of this constructed triangle lies on the circle using the converse of the alternating quadrilateral theorem.
	\item Prove that the triangle formed by $A$, $B$, and the added point is equilateral.
\end{enumerate}

\textbf{Solution 1:}

\begin{figure}[H]
	\begin{center}
		\hfill
		\begin{tikzpicture}[scale=3]
			\coordinate (A) at (0.25, 0.9682458365518544);
\coordinate (B) at (-0.7135254915624212, -0.7006292692220368);
\coordinate (C) at (1.2135254915624212, -0.7006292692220368);
\coordinate (D) at (0.7135254915624212, -0.7006292692220368);
\coordinate (E) at (-0.9635254915624212, -0.26761656732981753);
\coordinate (F) at (0.9635254915624212, 0.26761656732981753);
\coordinate (a) at (0.4817627457812106, 0.13380828366490877);
\coordinate (b) at (0.3658813728906053, 0.5510270601083815);
\coordinate (c) at (0.125, 0.4841229182759272);
\coordinate (d) at (0.5590169943749475, -0.14433756729740643);
	
			\node at (A) [above = 0.1 of A] {$A$};
			\node at (B) [below left = 0.1 of B] {$B$};
			\node at (C) [below right = 0.1 of C] {$C$};
			\node at (D) [below right = 0.1 of D] {$D$};
			\node at (E) [left = 0.1 of E] {$E$};
			\node at (a) [left = 0.2 of a] {$9$};
			
			\draw (0, 0) circle[radius=1];
			
			\draw[fill=blue, fill opacity=0.2] (A) -- (D) -- (C) -- (A);
			\draw[fill=blue, fill opacity=0.2] (A) -- (E) -- (B) -- (A);
			\draw (B) -- (D);
			
			\tkzMarkSegment[pos=0.5, mark=|](A, E);
			\tkzMarkSegment[pos=0.5, mark=|](A, D);
			\tkzMarkSegment[pos=0.5, mark=||](A, B);
			\tkzMarkSegment[pos=0.5, mark=||](A, C);
			
			\pic[draw, angle radius=30, "$60\degree$"] {angle = D--B--A};
			\pic[draw, angle radius=30, "$60\degree$"] {angle = A--C--D};
			\pic[draw, angle radius=30, "$60\degree$"] {angle = A--B--E};
			\pic[draw, angle radius=30, angle eccentricity=1.3, "$\alpha$"] {angle = D--A--C};
			\pic[draw, angle radius=30, angle eccentricity=1.3, "$\alpha$"] {angle = E--A--B};
			\pic[draw, angle radius=30, angle eccentricity=1.3, "$\beta$"] {angle = B--A--D};
			
			\label{fig:AlmostInscribedTriangle_CyclicQuadrilateral}
		\end{tikzpicture}
		\hfill
		\begin{tikzpicture}[scale=3]
			\coordinate (A) at (0.25, 0.9682458365518544);
\coordinate (B) at (-0.7135254915624212, -0.7006292692220368);
\coordinate (C) at (1.2135254915624212, -0.7006292692220368);
\coordinate (D) at (0.7135254915624212, -0.7006292692220368);
\coordinate (E) at (-0.9635254915624212, -0.26761656732981753);
\coordinate (F) at (0.9635254915624212, 0.26761656732981753);
\coordinate (a) at (0.4817627457812106, 0.13380828366490877);
\coordinate (b) at (0.3658813728906053, 0.5510270601083815);
\coordinate (c) at (0.125, 0.4841229182759272);
\coordinate (d) at (0.5590169943749475, -0.14433756729740643);
			
			\node at (A) [above = 0.1 of A] {$A$};
			\node at (B) [below left = 0.1 of B] {$B$};
			\node at (C) [below right = 0.1 of C] {$C$};
			\node at (D) [below right = 0.1 of D] {$D$};
			\node at (E) [left = 0.1 of E] {$E$};
			\node at (a) [left = 0.2 of a] {$9$};
			
			\draw (0, 0) circle[radius=1];
			\draw[fill=red, fill opacity=0.2] (A) -- (E) -- (D) -- (A);
			\draw (A) -- (B);
			\draw (A) -- (C);
			\draw (B) -- (D);
			\draw (B) -- (E);
			\draw (C) -- (D);
			
			\tkzMarkSegment[pos=0.5, mark=|](A, E);
			\tkzMarkSegment[pos=0.5, mark=|](A, D);
			\tkzMarkSegment[pos=0.5, mark=|](E, D);
			
			\pic[draw, angle radius=30, angle eccentricity=1.3, "$60\degree$"] {angle = E--A--D};
			\label{fig:AlmostInscribedTriangle_EquilaterialTriangle}
		\end{tikzpicture}
		\hfill
	\end{center}
	\caption{Solution 1}
\end{figure}

Noting that $|AB| = |AC|$, we construct triangle $EBA$ to be congruent to triangle $DCA$ as shown highlighted in figure~\ref{fig:AlmostInscribedTriangle_CyclicQuadrilateral}. Our aim is to show that the point added to form this triangle, $E$, lies on the circle. Defining $\alpha = \angle BAE$ and $\beta = \angle DAB$, we find $\angle DAE$ by equation~\eqref{eqn:AlmostInscribedTriangleInCircle_AngleDAE}.

\begin{equation}
	\angle DAE = \angle DAB + \angle BAE = \beta + \alpha = \alpha + \beta = \angle CAD + \angle DAB = 60\degree
	\label{eqn:AlmostInscribedTriangleInCircle_AngleDAE}
\end{equation}

By congruence of triangles $EBA$ and $DCA$, $\angle EBA = \angle DCA = 60\degree$, and thus $\angle EBD = 120\degree$. Therefore $\angle EBD$ and $\angle DAE$, opposite angles of quadrilateral $AEBD$, add to $180\degree$. As $\angle AEB$ and $\angle ADC$ are equal by congruence of triangles $EBA$ and $DCA$ and $\angle BDA$ and $\angle ADC$ make the straight line $BDC$, we deduce that opposite angles of quadrilateral $AEBD$ ($\angle AEB$ and $\angle BDA$) add to $180\degree$ as well. Therefore quadrilateral $AEBD$ is a cyclic quadrilateral and thus point $E$ lies on the circle.

By definition $|AD| = |AE|$ which makes triangle $DAE$ isocoles. As shown before, $\angle DAE = 60\degree$, therefore $\angle AED + \angle EDA = 120\degree$. As these angles are equal from triangle $DAE$ being isoceles they must both be $60\degree$, and thus triangle $DAE$ is in fact equilateral. By rotational symmetry the centre of of the triangle must be the centre of the circle. With the simple construction given in figure~\ref{fig:AlmostInscribedTriangle_FindingArea} and basic trigonometry we see that the area of the circle is $27\pi$ as shown in equation~\eqref{eqn:AlmostInscribedTriangleInCircle_ComputingRadiusAndArea}.

\begin{equation}
	r = \frac{4.5}{\sin(60\degree)} = \frac{4.5}{\frac{\sqrt{3}}{2}} = 3\sqrt{3} \implies \text{Area} = \pi r^2 = 27\pi
	\label{eqn:AlmostInscribedTriangleInCircle_ComputingRadiusAndArea}
\end{equation}

\textbf{Hints 2:}

\begin{enumerate}
	\item Find an angle at the centre of the circle.
	\item Draw a kite connecting the circle centre, $A$, $B$, and $D$.
	\item The angle subtended by an arc at the circle centre is double the angle subtended by the same arc at the circle edge.
\end{enumerate}

\textbf{Solution 2:}

Connect the circle centre, $O$, to points $A$ and $D$ to form a kite, as shown in figure~\ref{fig:AlmostInscribedTriangle_Kite} below. As the angle subtended by an arc at the centre is double the angle subtended at the circle edge we get $\angle DOA = 2 \cdot \angle DBA = 120\degree$ and the solution continues identically to solution 1.

\begin{figure}[H]
	\begin{center}
		\hfill
		\begin{tikzpicture}[scale=3]
			\coordinate (A) at (0.25, 0.9682458365518544);
\coordinate (B) at (-0.7135254915624212, -0.7006292692220368);
\coordinate (C) at (1.2135254915624212, -0.7006292692220368);
\coordinate (D) at (0.7135254915624212, -0.7006292692220368);
\coordinate (E) at (-0.9635254915624212, -0.26761656732981753);
\coordinate (F) at (0.9635254915624212, 0.26761656732981753);
\coordinate (a) at (0.4817627457812106, 0.13380828366490877);
\coordinate (b) at (0.3658813728906053, 0.5510270601083815);
\coordinate (c) at (0.125, 0.4841229182759272);
\coordinate (d) at (0.5590169943749475, -0.14433756729740643);
			
			\coordinate (O) at (0, 0);
			
			\node at (A) [above = 0.1 of A] {$A$};
			\node at (B) [below left = 0.1 of B] {$B$};
			\node at (C) [below right = 0.1 of C] {$C$};
			\node at (D) [below right = 0.1 of D] {$D$};
			\node at (O) [left = 0.15 of O] {$O$};
			\node at (d) [right = 0.15 of d] {$9$};
			
			\draw (O) circle[radius=1];
			\draw (A) -- (C);
			\draw (C) -- (D);
			\draw (A) -- (D);
			\draw[fill=blue, fill opacity=0.2] (A) -- (B) -- (D) -- (O) -- (A);
			
			\pic[draw, angle radius=30, "\hspace{2mm}$60\degree$"] {angle = D--B--A};
			\pic[draw, angle radius=30, "$120\degree$"] {angle = D--O--A};
			
			\label{fig:AlmostInscribedTriangle_Kite}
		\end{tikzpicture}
		\hfill
		\begin{tikzpicture}[scale=3]
			\coordinate (A) at (0.25, 0.9682458365518544);
\coordinate (B) at (-0.7135254915624212, -0.7006292692220368);
\coordinate (C) at (1.2135254915624212, -0.7006292692220368);
\coordinate (D) at (0.7135254915624212, -0.7006292692220368);
\coordinate (E) at (-0.9635254915624212, -0.26761656732981753);
\coordinate (F) at (0.9635254915624212, 0.26761656732981753);
\coordinate (a) at (0.4817627457812106, 0.13380828366490877);
\coordinate (b) at (0.3658813728906053, 0.5510270601083815);
\coordinate (c) at (0.125, 0.4841229182759272);
\coordinate (d) at (0.5590169943749475, -0.14433756729740643);
			
			\coordinate (O) at (0, 0);
			
			\node at (A) [above = 0.1 of A] {$A$};
			\node at (D) [below right = 0.1 of D] {$D$};
			\node at (O) [left = 0.15 of O] {$O$};
			\node at (F) [right = 0.15 of F] {$F$};
			\node at (b) [right = 0.1 of b] {$4.5$};
			\node at (c) [left = 0.1 of c] {$r$};
			
			\draw (O) circle[radius=1];
			\draw (A) -- (D);
			\draw (O) -- (A);
			\draw (O) -- (D);
			\draw (O) -- (F);
			
			\pic[draw, angle radius=30, "\hspace{2mm}$60\degree$"] {angle = F--O--A};
			\pic[draw, angle radius=10] {right angle = A--a--O};
			
			\label{fig:AlmostInscribedTriangle_FindingArea}
		\end{tikzpicture}
		\hfill
	\end{center}
	\caption{Solution 2 and the common method for finding the circle area}
\end{figure}

\textbf{Extensions and Comments:}

I tried solving this while on the train and was not bothered to get my whiteboard out. I managed to find and prove the cyclic quadrilateral, but did not spot the isocoles triangle which would have made the problem trivial.
