\subsection{The Snails on a Plank Problem}

This problem appears very difficult at first, but has a very simple solution that I personally think is rather unintuitive. I've asked several people this problem as of writing this, and all of them fell right into the trap which pulled them away from the true solution. I cannot remember where I got this problem from, and rate this problem 4/10 in terms of difficulty. This is too high a rating given the simplicity of the solution, but it really seems to throw people off, even when the solution is explained to them.

There are $n$ snails on a plank which can be assumed to be one dimensional. The snails are randomly distributed on the plank and can be approximated at point particles. They all move at the same speed but in random directions, either left or right along the plank. When they get to the end they fall off, and they swap directions when running into another snail. It takes a single snail one minute to go from one end of the plank to another. How long does it take for all snails to fall off the plank, and more importantly why is that the answer? Extension: if all left-going snails start to the right of the right-going snails, how many times do snails run into each other?

\textbf{Hints:}

\begin{enumerate}
	\item The answer is independent of the number of snails and the starting configuration.
	\item It takes at most one minute before all snails have fallen off the plank.
	\item Consider the case of just two snails, but do not attempt to prove this by induction.
	\item Suppose a blue snail heading right collides with a red snail heading left. Both swap directions, but suppose they also swap colours. Visualise what this would look like.
\end{enumerate}

\textbf{Solution:}

As it may be suspected by the complete lack of information given, all the apparent complexity cancels out in some way and the answer is a constant one minute which most people can guess. The trick to understanding why is to look at what happens when two snails interact. Two snails hitting and instantly changing direction is indistinguishable from two snails phasing through each other. This can be demonstrated with two fingers, in one case where they move towards each other and pass by each other, and in the other case where they change directions. The fourth hint suggests this very strongly and is another good way to picture why this is the case.

Applying the reasoning of two snails interacting to all the snails, it would look like all the snails heading right do so unimpeded, and the same for the snails heading left. They simply phase through each other and can therefore be treated individually, thus reducing to the case for one snail which takes at most one minute. Using this, the general case of the extension can be easily seen as the sum of the number of left-going snails to the right of each right-going snail. Where all right-going snails start to the left of the left-going snails, this is the product of the left-going and right-going snails.

\textbf{Extensions and Comments:}

What I believe causes difficulty in this problem is thinking about the snails as individuals. When considering a pair, the answer is almost immediate, but it seems most natural to think of one snail at a time. My dad attempted to program this and I told him one of two things would happen. He would either finish programming it and think there was a bug when it first ran (i.e. expecting to see snails bounding off each other but actually seeing them phase through each other), or he would realise the solution before finishing (when coding a snail collision). The latter of these is exactly what happened.

Many people start off with one and then two snails, but make the mistake of considering three or more. A commonly considered scenario would be to place one snail at the end and another halfway along, and note the snails would travel $0.25 + 0.25 = 0.5$ and $0.25 + 0.75 = 1$ times the length of the plank respectively. This convinced people it was one minute, but took them no further. I made a Desmos simulating this that shows the behaviour: \url{https://www.desmos.com/calculator/g2mcmsmc1m}.

Interestingly, it is impossible for two snails to collide after 30 seconds, or equivalently after any snail had travelled a total distance of halfway along the plank. This is a consequence of the fact that the snails must fall off by the end of the minute. Explicitly, a snail has to be within $1 - t$ of the end of the plank it would eventually fall off at time $t$, where $t$ is the time in minutes. If two snails were on the left side after 30 seconds, then they cannot reach the right side. They must be travelling in opposite directions in order to collide, meaning one would go left and the other would go right. The right-going snail would not reach the right side in time. They cannot be saved by colliding with a left-going, as this simply kicks the snail down the road, as the other snail would take their place.

The problem is phrased rather imprecisely, where the exact question that is sought is actually the supremum of times it takes for snails to fall of. We can go further than this however, and find the probability distribution of times for the last snail to fall off. Let $T_i$ be the time taken for the $i^\text{th}$ snail to fall off, and $T = \max_i{T_i}$ be the time for the last snail to fall off. The labelling of the snails does not matter as we are only interested in $T$, and we can focus on the equivalent situation where snails phase through each other. The time it takes for the snails to reach the end is exactly proportional to the distance from the end they will fall off. As the snails are uniformly distributed spatial, the $T_i$ are also uniformly distributed, giving $T_i \sim U(0, 1)$ as a set of independent and identically distributed random variables. The cumulative distribution of $T$ can be found as follows.

\begin{align*}
    \mathbb{P}(T \le t) &= 1 - \mathbb{P}(T \ge t)  \\
    &= 1 - \prod_{i=1}^n \mathbb{P}(T_i \ge t)  \\
    &= 1 - \prod_{i=1} \left(1 - \mathbb{P}(T_i \le t)\right)  \\
    &= 1 - \prod_{i=1}^n (1 - t)  \\
    &= 1 - (1 - t)^n
\end{align*}