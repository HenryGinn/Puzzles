\subsection{The Chessboard Encoding Problem}

This problem is from a collaboration between 3blue1brown and Matt Parker. I attempted this problem years ago, but could not do it, but on a second attempt recently I got it instanteously. I think I spoiled this problem for myself however as I had watched a video recently about a highly relevant problem, and I would never have got it so fast otherwise. I give this problem a difficulty of 5/10, although the ideas behind the solution did lead to someone winning the Turing Prize.

You and a colleague are inprisoned, although the prison warden gives you an opportunity for freedom. There is an $8 \times 8$ chessboard where each square is a hidden compartment, and the warden reveals to one of you, let us call them person A, which square the key to the door is hidden in. A coin is then placed on each square in a configuration decided by the warden that is not known to you or your colleague in advance. Person A has to turn over exactly one coin before leaving the room, at which point person B enters and needs to determine which square has the key. You are allowed to agree a strategy beforehand, although the warden will hear your strategy, and will try and find a configuration of coins to make your strategy not work.

\textbf{Hints:}

\begin{enumerate}
	\item Consider a board with two squares, then four squares.
	\item There exists a reliable strategy if and only if the number of squares is a power of two~\cite{}.
	\item Consider the parity of the number of coins that are heads-up in a region.
\end{enumerate}

\textbf{Solution:}

The solution works by being able to control the parity of the number of heads-up coins in six regions. This produces a six-digit binary number which can be used to encode a square of the chessboard as $8 \times 8 = 64 = 2^6$. The key is to be able to control the parity in each region indepently, and the six regions shown in figure~\ref{fig:TheChessboardEncodingProblem_Grids} allow this.

\begin{figure}[H]
	\begin{center}
		\begin{tikzpicture}[scale=0.6]
			\input{1 - Logic/TheChessboardEncodingProblem/Chessboard_1.tikz}
		\end{tikzpicture}\hspace{5mm}
		\begin{tikzpicture}[scale=0.6]
			\input{1 - Logic/TheChessboardEncodingProblem/Chessboard_2.tikz}
		\end{tikzpicture}\hspace{5mm}
		\begin{tikzpicture}[scale=0.6]
			\input{1 - Logic/TheChessboardEncodingProblem/Chessboard_3.tikz}
		\end{tikzpicture}\vspace{6mm}
		\\
		\begin{tikzpicture}[scale=0.6]
			\input{1 - Logic/TheChessboardEncodingProblem/Chessboard_4.tikz}
		\end{tikzpicture}\hspace{5mm}
		\begin{tikzpicture}[scale=0.6]
			\input{1 - Logic/TheChessboardEncodingProblem/Chessboard_5.tikz}
		\end{tikzpicture}\hspace{5mm}
		\begin{tikzpicture}[scale=0.6]
			\input{1 - Logic/TheChessboardEncodingProblem/Chessboard_6.tikz}
		\end{tikzpicture}
	\end{center}
	\caption{Six regions that each define a binary bit via the parity of the number of heads-up coins in the squares highlighted in red.}
	\label{fig:TheChessboardEncodingProblem_Grids}
\end{figure}

In the example shown in figure~\ref{fig:TheChessboardEncodingProblem_Example} the key has been hidden in fourth row and third column. Using the first three digits to encode the column number and the last three digits to encode the row number, this is encoded as 011100. In the red highlighted region in figure~\ref{fig:TheChessboardEncodingProblem_1} there are 16 heads which is equal to 0 modulo 2. This matches the first digit in the number we want to encode which means we want to preserve the parity of heads in this region, therefore we will flip a coin outside the red highlighted region, that is, on the left. We can do a similar process for the other regions and this is described in table~\ref{tab:TheChessboardEncodingProblem_ExampleParity}

\begin{figure}[H]
	\centering
	\begin{tikzpicture}[scale=1]
		\input{1 - Logic/TheChessboardEncodingProblem/ChessboardExample.tikz}
	\end{tikzpicture}
	\caption{An example of random coins on a chessboard. The red highlighted square indicates the position of the key.}
	\label{fig:TheChessboardEncodingProblem_Example}
\end{figure}

\begin{table}[H]
	\centering
	\caption{An example of how to determine which regions to flip the parity of}
	\label{tab:TheChessboardEncodingProblem_ExampleParity}
	\begin{tabular}{cccccc}
		\myhline
		Region & Head Count & Head Parity & Encoded Parity & Parity Change & Flip Red Region  \\
		\myhline
		\tableinput{1 - Logic/TheChessboardEncodingProblem/ChessboardExampleTable.tex}
		\myhline
	\end{tabular}
\end{table}

\textbf{Extensions and Comments:}
