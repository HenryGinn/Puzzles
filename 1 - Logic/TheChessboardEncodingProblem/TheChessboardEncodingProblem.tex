\subsection{The Chessboard Encoding Problem}

This problem is from a collaboration between 3blue1brown and Matt Parker. I attempted this problem years ago, but could not do it, but on a second attempt recently I got it instanteously. I think I spoiled this problem for myself however as I had watched a video recently about a highly relevant problem, and I would never have got it so fast otherwise. I give this problem a difficulty of 5/10, although the ideas behind the solution did lead to someone winning the Turing Prize.

You and a colleague are inprisoned, although the prison warden gives you an opportunity for freedom. There is an $8 \times 8$ chessboard where each square is a hidden compartment, and the warden reveals to one of you, let us call them person A, which square the key to the door is hidden in. A coin is then placed on each square in a configuration decided by the warden that is not known to you or your colleague in advance. Person A has to turn over exactly one coin before leaving the room, at which point person B enters and needs to determine which square has the key. You are allowed to agree a strategy beforehand, although the warden will hear your strategy, and will try and find a configuration of coins to make your strategy not work.

\textbf{Hints:}

\begin{enumerate}
	\item Consider a board with two squares, then four squares.
	\item There exists a reliable strategy if and only if the number of squares is a power of two~\cite{}.
	\item Consider the parity of the number of coins that are heads-up in a region.
\end{enumerate}

\textbf{Solution:}

The solution works by being able to control the parity of the number of heads-up coins in six regions. This produces a six-digit binary number which can be used to encode the square of the chessboard as $8 \times 8 = 64 = 2^6$. The key is to be able to control the parity in each region indepently, and the six regions shown in figure~\ref{} allow this.

\begin{figure}[H]
	\begin{center}
		\begin{tikzpicture}[scale=0.6]
			\input{1 - Logic/TheChessboardEncodingProblem/Chessboard_1.tikz}
		\end{tikzpicture}\hspace{5mm}
		\begin{tikzpicture}[scale=0.6]
			\input{1 - Logic/TheChessboardEncodingProblem/Chessboard_2.tikz}
		\end{tikzpicture}\hspace{5mm}
		\begin{tikzpicture}[scale=0.6]
			\input{1 - Logic/TheChessboardEncodingProblem/Chessboard_3.tikz}
		\end{tikzpicture}
		\vspace{5mm}\\
		\begin{tikzpicture}[scale=0.6]
			\input{1 - Logic/TheChessboardEncodingProblem/Chessboard_4.tikz}
		\end{tikzpicture}\hspace{5mm}
		\begin{tikzpicture}[scale=0.6]
			\input{1 - Logic/TheChessboardEncodingProblem/Chessboard_5.tikz}
		\end{tikzpicture}\hspace{5mm}
		\begin{tikzpicture}[scale=0.6]
			\input{1 - Logic/TheChessboardEncodingProblem/Chessboard_6.tikz}
		\end{tikzpicture}
	\end{center}
\end{figure}

\textbf{Extensions and Comments:}






















